\documentclass[12pt]{article}

% Packages
\usepackage{setspace}
\doublespacing

\usepackage[margin=1in]{geometry} % Set the margins to 1 inch

\usepackage{lineno} % For line numbers
\linenumbers

% Packages
\usepackage{amsmath} % For mathematical symbols and equations
\usepackage{amssymb} % For additional mathematical symbols
\usepackage{graphicx} % For including figures
\usepackage{cite} % For citations
\usepackage{setspace} % For line spacing
\usepackage{lipsum} % For placeholder text (remove in your actual article)

% Title and authors
\title{Your Title Here}
\author{Author 1 \and Author 2 \and Author 3}
\date{} % Remove this line to display the current date

\begin{document}

% Title and abstract
\maketitle

\begin{abstract}
    Your abstract goes here.
\end{abstract}

% Main body
\section{Introduction}
\label{sec:introduction}

\subsection{Background and Motivation}
\label{subsec:intro_Background}

% Background and motivation

% Sentence 1: Provide a general background of the research area or problem.

% Sentence 2: Highlight the significance or importance of the problem.

% Sentence 3: Identify any gaps or limitations in the current knowledge or existing solutions.

% Sentence 4: State the specific objective or research question addressed in your study.

% Sentence 5: Give a brief overview of how your research contributes to filling the gaps or addressing the problem.

\subsection{Clinical Literature Review}
\label{subsec:intro_clinical}

% Intro sentence: Provide an introduction to the literature review and its relevance to your study.

% Clinical literature review 1

% Sentence 1: Cite and briefly explain the first relevant study in the literature that demonstrates the clinical significance of your research topic.

% Sentence 2: Discuss the significance or contribution of this study to the field of medical imaging or the specific clinical application you are targeting.

% Sentence 3: Identify the limitations or challenges of the study that your research aims to address or overcome.

% Literature review 2

% Sentence 1: Cite and briefly explain the second relevant study in the literature that supports the clinical relevance of your research topic.

% Sentence 2: Discuss the significance or impact of this study on the research area or clinical practice.

% Sentence 3: Outline the limitations or gaps in the existing methods or technologies that your research aims to overcome or improve upon.

% Conclusion sentence: Summarize the importance of the reviewed clinical literature and indicate how your research bridges the identified gaps or addresses the limitations.


\subsection{Engineering Literature Review}
\label{subsec:intro_engineering}

% Literature review 1

% Sentence 1: Cite and briefly explain the third relevant study in the literature that describes the existing methods or techniques related to your research topic.

% Sentence 2: Explain the significance or implications of this study for your research and how it has influenced the field or clinical practice.

% Sentence 3: Highlight the specific limitations or drawbacks of the study's methods or techniques that your research aims to address or overcome.

% Literature review 2

% Sentence 1: Cite and briefly explain the fourth relevant study in the literature that describes additional existing methods or approaches in your research area.

% Sentence 2: Explain the significance or implications of this study for your research and how it complements or relates to the previous methods discussed.

% Sentence 3: Highlight the specific limitations or gaps in the methods or approaches described in this study that your research aims to tackle or improve upon.

% Conclusion sentence: Summarize the key findings of the literature review on existing methods, emphasizing the need for further advancements or novel approaches, which your research aims to fulfill.


\subsection{Outline of Methods, Results, and Conclusions}
\label{subsec:intro_outline}

Clearly state the research objectives and outline the structure of your paper.

% Methods section
\section{Methods}
\label{sec:methods}

% Subsection 1: Summary of Approach
\subsection{Outline of Methods}
\label{subsec:methods_outline}

Provide a brief overview of your research approach and methodology.

% Subsection 2: Theoretical Methods
\subsection{Theoretical Methods}
\label{subsec:theoretical}

Explain the theoretical foundations or models used in your research.

% Subsection 3: Experimental Methods
\subsection{Experimental Methods}


\label{subsec:experimental}

Describe the experimental setup, materials, and procedures employed.

% Subsection 4: Analysis Methods
\subsection{Analysis Methods}
\label{subsec:analysis}

Detail the data analysis techniques or statistical methods used in your research.

% Results section
\section{Results}
\label{sec:results}

% Subsection 1: Results Subsection 1
\subsection{Results Subsection 1}
\label{subsec:results1}

Present the findings of your research related to Subsection 1.

% Subsection 2: Results Subsection 2
\subsection{Results Subsection 2}
\label{subsec:results2}

Present the findings of your research related to Subsection 2.

% Subsection 3: Results Subsection 3
\subsection{Results Subsection 3}
\label{subsec:results3}

Present the findings of your research related to Subsection 3.

% Conclusion section
\section{Conclusion}
\label{sec:conclusion}

% Subsection 1: Conclusion Subsection 1
\subsection{Conclusion Subsection 1}
\label{subsec:conclusion1}

Summarize your findings and their significance.

% Subsection 2: Conclusion Subsection 2
\subsection{Conclusion Subsection 2}
\label{subsec:conclusion2}

Discuss the implications of your results and their impact on the field.

% Subsection 3: Conclusion Subsection 3
\subsection{Conclusion Subsection 3}
\label{subsec:conclusion3}

Highlight the limitations of your study and potential areas for improvement.

% Subsection 4: Conclusion Subsection 4
\subsection{Conclusion Subsection 4}
\label{subsec:conclusion4}

Provide concluding remarks, future directions, and suggestions for further research.

% Acknowledgments (optional)
\section*{Acknowledgments}

Include any acknowledgments here.

% References
\bibliographystyle{plain}
\bibliography{references}

\end{document}