\documentclass[12pt]{article}

% Packages
\usepackage{setspace}
\doublespacing

\usepackage[margin=1in]{geometry} % Set the margins to 1 inch

\usepackage{lineno} % For line numbers
\linenumbers

% Packages
\usepackage{amsmath} % For mathematical symbols and equations
\usepackage{amssymb} % For additional mathematical symbols
\usepackage{graphicx} % For including figures
\usepackage{cite} % For citations
\usepackage{setspace} % For line spacing
\usepackage{lipsum} % For placeholder text (remove in your actual article)

% Title and authors
\title{Your Title Here}
\author{Author 1 \and Author 2 \and Author 3}
\date{} % Remove this line to display the current date

\begin{document}

% Title and abstract
\maketitle

\begin{abstract}
    Your abstract goes here.
\end{abstract}

% Main body
\section{Introduction}
\label{sec:introduction}

\subsection{Background and Motivation}
\label{subsec:intro_Background}

% Background and motivation

% Sentence 1: Highlight the significance or importance of the problem.

Pancreatic cancer is the fourth leading cause of cancer-related deaths in the United States, with a five-year survival rate of only 9\% \cite{Siegel2019}.

% Sentence 2: Explain the clinical motivation for new technology

Early detection of pancreatic cancer is critical for improving patient outcomes, as the five-year survival rate increases to 34\% when the cancer is detected at an early stage \cite{Siegel2019}.

% Sentence 3: Identify the new technology that can be used to address the problem

Iodine-contrast-enhanced dual-energy CT DECT is a promising imaging modality for pancreatic cancer detection, as it can provide quantitative information about tissue composition and function \cite{Johnson2017}.

DECT technologies such as dual-source or dual-layer detectors have enabled quantitative iodine concentration imaging. 

% Sentence 4: Identify any gaps or limitations in the current knowledge or existing solutions.

However, the concentration of iodine in the pancreas is very low compared to other organs, which makes it difficult to detect early-stage pancreatic tumors.

% Sentence 5: State the specific objective or research question addressed in your study.

The objective of this study to investigate hybrid spectral CT system designs that combine multiple dual-energy techniques in a single imaging system.

% Sentence 6: Give a brief overview of how your research contributes to filling the gaps or addressing the problem.

We believe that such hybrid systems can reduce noise and bias in quantitative iodine concentration imaging which could potentially improve the detection of early-stage pancreatic tumors.

\subsection{Clinical Significance}
\label{subsec:intro_clinical}

% Intro sentence: Provide an introduction to the literature review and its relevance to your study.

% Clinical literature review 1

% Literature review 2

% Sentence 1: Cite and briefly explain the first relevant study in the literature that supports the clinical relevance of your research topic.

One major challenge for imaging pancreatic cancert with iodine-enhanced DECT or spectral CT is that iodine contrast agent uptake in the pancreas is often lower than other organs, due to the fibrous tissue within and surrounding the organ \cite{Johnson2017}.

This is especially true at early stages of the disease \cite{Johnson2017}.

% Sentence 2: Discuss the significance or impact of this study on the research area or clinical practice.

Such low-contrast features are challenging to detect for existing DECT systems because the contrast-to-noise ratio is low. 

% Sentence 1: Cite and briefly explain the second relevant study in the literature that demonstrates the clinical significance of your research topic.

One of the main reasons that pancreatic cancer is so deadly is that it is often not detected until it has reached an advanced stage \cite{Siegel2019}.

% Sentence 2: Discuss the significance or contribution of this study to the field of medical imaging or the specific clinical application you are targeting.

If pancreatic cancer can be detected at an earlier stage, the average prognosis for patients could be improved.

% Conclusion sentence: Summarize the importance of the reviewed clinical literature and indicate how your research bridges the identified gaps or addresses the limitations.

New technologies are urgently needed to reduce noise and bias in quantitative iodine concentration imaging with DECT or spectral CT. 

\subsection{Engineering Background}
\label{subsec:intro_engineering}

% Intro sentence: Provide an introduction to the literature review and its relevance to your study.

The existing technologies for DECT and spectral CT can be divided into source-based and detector-based techniques.

% Literature review 1

% Sentence 1: Cite and briefly explain the first relevant study in the literature that describes the existing methods or techniques related to your research topic.

Source-based techniques use x-ray sources with two or more transmission spectra. 

Examples include dual-source CT, which uses two x-ray tubes with different peak-kilo-voltage (kVp) settings or filters, rapid kVp switching, which uses a single x-ray tube with two different kVp settings, and structured source filtration techniques, such as split filters and spatial-spectral filters. 

% Sentence 2: Explain the significance or implications of this study for your research and how it has influenced the field or clinical practice.

By using multiple transmission spectra, source-based techniques can provide information about the energy-dependent attenuation properties of the object, which can be used to estimate the object's material composition.

% Sentence 1: Cite and briefly explain the second relevant study in the literature that describes additional existing methods or approaches in your research area.

There are also detector-based techniques that that use energy-discriminating detectors (EDDs) to measure the x-ray spectrum transmitted through the object.

Examples include dual-layer detectors, which use two layers of scintillators with the option for additional interstitial filtration, and photon-counting detectors, which use semiconductor materials to estimate the energy of individual measured x-ray photons.

% Sentence 2: Explain the significance or implications of this study for your research and how it complements or relates to the previous methods discussed.

EDDs enable material decomposition and quantitative iodine imaging with a single x-ray source without kVp switching or filtration . 

% Sentence 3: Highlight the specific limitations or gaps in the methods or approaches described in this study that your research aims to tackle or improve upon.

Moreover, these EDDs can be used together with one or more source-based spectral CT techniques to improve material separability and image quality. 

% Conclusion sentence: Summarize the key findings of the literature review on existing methods, emphasizing the need for further advancements or novel approaches, which your research aims to fulfill.

In this work, we investigate the hybrid spectral CT systems that combine multiple source-based and detector-based techniques to reduce noise and bias in quantitative iodine concentration imaging.


\subsection{Innovation}
\label{subsec:intro_innovation}




% Intro sentence: Acknowledge the existence of related research in your field and explain how your research builds upon or extends the existing knowledge.

Previous studies have explored the use of hybrid spectral CT reconstruction techniques by combining energy discriminating detectors (EDDs), specifically photon-counting detectors (PCDs), and energy integrating detectors (EIDs) into a joint reconstruction process [cite].

% Literature review 1

% Sentence 1: Cite and briefly explain the first relevant study in the literature that describes the existing methods or techniques related to your research topic.

For instance, in \cite{clark2017hybrid}, the authors proposed a hybrid spectral CT reconstruction method that incorporated PCDs and EIDs to address limitations associated with spectral fidelity and photon starvation in PCDs. 

Their approach aimed to combine the spectral properties of PCD data with the resolution and signal-to-noise characteristics of EID data.

% Sentence 2: Explain the significance or implications of this study for your research and how it has influenced the field or clinical practice.

By leveraging the strengths of PCDs and EIDs, this work demonstrated the potential to improve the quality of spectral CT images and enhance the detection and separation of specific materials.

However, their approach relied on data acquired from two independent detection systems, so the total number of spectral channels is the addition of spectral channels in each acquisition.

% Sentence 3: Highlight the specific limitations or drawbacks of the study's methods or techniques that your research aims to address or overcome.

In contrast, our research focuses on simulating joint acquisitions that combine EDDs with source-based spectral technologies in the same scan, resulting in a multiplication of the number of spectral channels rather than addition. 

%  Literature review 2





\subsection{Outline of Methods, Results, and Conclusions}
\label{subsec:intro_outline}

Clearly state the research objectives and outline the structure of your paper.

% Methods section
\section{Methods}
\label{sec:methods}

% Subsection 1: Summary of Approach
\subsection{Outline of Methods}
\label{subsec:methods_outline}

Provide a brief overview of your research approach and methodology.

% Subsection 2: Theoretical Methods
\subsection{Theoretical Methods}
\label{subsec:theoretical}

Explain the theoretical foundations or models used in your research.

% Subsection 3: Experimental Methods
\subsection{Experimental Methods}


\label{subsec:experimental}

Describe the experimental setup, materials, and procedures employed.

% Subsection 4: Analysis Methods
\subsection{Analysis Methods}
\label{subsec:analysis}

Detail the data analysis techniques or statistical methods used in your research.

% Results section
\section{Results}
\label{sec:results}

% Subsection 1: Results Subsection 1
\subsection{Results Subsection 1}
\label{subsec:results1}

Present the findings of your research related to Subsection 1.

% Subsection 2: Results Subsection 2
\subsection{Results Subsection 2}
\label{subsec:results2}

Present the findings of your research related to Subsection 2.

% Subsection 3: Results Subsection 3
\subsection{Results Subsection 3}
\label{subsec:results3}

Present the findings of your research related to Subsection 3.

% Conclusion section
\section{Conclusion}
\label{sec:conclusion}

% Subsection 1: Conclusion Subsection 1
\subsection{Conclusion Subsection 1}
\label{subsec:conclusion1}

Summarize your findings and their significance.

% Subsection 2: Conclusion Subsection 2
\subsection{Conclusion Subsection 2}
\label{subsec:conclusion2}

Discuss the implications of your results and their impact on the field.

% Subsection 3: Conclusion Subsection 3
\subsection{Conclusion Subsection 3}
\label{subsec:conclusion3}

Highlight the limitations of your study and potential areas for improvement.

% Subsection 4: Conclusion Subsection 4
\subsection{Conclusion Subsection 4}
\label{subsec:conclusion4}

Provide concluding remarks, future directions, and suggestions for further research.

% Acknowledgments (optional)
\section*{Acknowledgments}

Include any acknowledgments here.

% References
\bibliographystyle{plain}
\bibliography{references}

\end{document}